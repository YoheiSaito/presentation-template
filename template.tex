\documentclass[dvipdfmx, a4p, cjk]{beamer}

% template package
\usepackage{txfonts}
\usepackage{hyperref}
\usepackage{pxjahyper}
\usepackage{bxdpx-beamer}
\usepackage{color}
\usetheme{Madrid}
\usecolortheme{dolphin}

\usefonttheme{professionalfonts}
\setbeamertemplate{navigation symbols}{} % remove navigation
% \setbeamertemplate{footline}[page number]

\newcommand{\argmax}{\mathop{\rm arg~max}\limits}
\newcommand{\argmin}{\mathop{\rm arg~min}\limits}
\newcommand{\red}[1]{\textcolor{red}{#1}}
\newcommand{\green}[1]{\textcolor{green}{#1}}
\newcommand{\blue}[1]{\textcolor{blue}{#1}}
\newcommand{\cyan}[1]{\textcolor{cyan}{#1}}
\newcommand{\magenta}[1]{\textcolor{magenta}{#1}}
\newcommand{\yellow}[1]{\textcolor{yellow}{#1}}
\title{2020年度第1回 \\ 点群進捗報告会}
\author{齊藤 陽平}
\institute[田中研究室]{信州大学総合理工学研究科 田中エルナン宮川研究室}
\date{2020年 6月27日}

\begin{document}
\frame{\titlepage \thispagestyle{empty}}
\begin{frame}{あらすじ} \tableofcontents \end{frame}
\begin{frame}{フレームタイトル}
内容をここに書く.

\blue{■} \hspace{-9pt} \cyan{■} \hspace{-9pt} \green{■} \hspace{-9pt} \yellow{■} \hspace{-9pt} \red{■} \hspace{-9pt} \magenta{■} \vspace{-6pt}\\
\blue{■} \hspace{-9pt} \cyan{■} \hspace{-9pt} \green{■} \hspace{-9pt} \yellow{■} \hspace{-9pt} \red{■} \hspace{-9pt} \magenta{■} \vspace{-6pt}\\
\blue{■} \hspace{-9pt} \cyan{■} \hspace{-9pt} \green{■} \hspace{-9pt} \yellow{■} \hspace{-9pt} \red{■} \hspace{-9pt} \magenta{■} \vspace{-6pt}\\
\blue{■} \hspace{-9pt} \cyan{■} \hspace{-9pt} \green{■} \hspace{-9pt} \yellow{■} \hspace{-9pt} \red{■} \hspace{-9pt} \magenta{■} \vspace{-6pt}\\
\blue{■} \hspace{-9pt} \cyan{■} \hspace{-9pt} \green{■} \hspace{-9pt} \yellow{■} \hspace{-9pt} \red{■} \hspace{-9pt} \magenta{■} \vspace{-6pt}\\
\blue{■} \hspace{-9pt} \cyan{■} \hspace{-9pt} \green{■} \hspace{-9pt} \yellow{■} \hspace{-9pt} \red{■} \hspace{-9pt} \magenta{■} \vspace{-6pt}\\
\blue{■} \hspace{-9pt} \cyan{■} \hspace{-9pt} \green{■} \hspace{-9pt} \yellow{■} \hspace{-9pt} \red{■} \hspace{-9pt} \magenta{■} \vspace{-6pt}\\
\blue{■} \hspace{-9pt} \cyan{■} \hspace{-9pt} \green{■} \hspace{-9pt} \yellow{■} \hspace{-9pt} \red{■} \hspace{-9pt} \magenta{■} \vspace{-6pt}\\
\blue{■} \hspace{-9pt} \cyan{■} \hspace{-9pt} \green{■} \hspace{-9pt} \yellow{■} \hspace{-9pt} \red{■} \hspace{-9pt} \magenta{■} \vspace{-6pt}\\
\blue{■} \hspace{-9pt} \cyan{■} \hspace{-9pt} \green{■} \hspace{-9pt} \yellow{■} \hspace{-9pt} \red{■} \hspace{-9pt} \magenta{■} \vspace{-6pt}\\
\blue{■} \hspace{-9pt} \cyan{■} \hspace{-9pt} \green{■} \hspace{-9pt} \yellow{■} \hspace{-9pt} \red{■} \hspace{-9pt} \magenta{■} \vspace{-6pt}\\
\blue{■} \hspace{-9pt} \cyan{■} \hspace{-9pt} \green{■} \hspace{-9pt} \yellow{■} \hspace{-9pt} \red{■} \hspace{-9pt} \magenta{■} \vspace{-6pt}\\
\blue{■} \hspace{-9pt} \cyan{■} \hspace{-9pt} \green{■} \hspace{-9pt} \yellow{■} \hspace{-9pt} \red{■} \hspace{-9pt} \magenta{■} \vspace{-6pt}\\
\blue{■} \hspace{-9pt} \cyan{■} \hspace{-9pt} \green{■} \hspace{-9pt} \yellow{■} \hspace{-9pt} \red{■} \hspace{-9pt} \magenta{■} \vspace{ 0pt}\\

\alert{強調}
\bf{bold}
\sl{naname}
\it{itaric}
数式:$a=b=c$

\end{frame}

\begin{frame}{はじめてのスライド}
    \begin{definition}
    1と自分自身しか約数を持たない数を\alert{素数}という.
    \end{definition}
    \begin{example}
        \begin{itemize}
        \item 2 は素数.
        \item 3 も素数.
        \item 4 は素数ではない.
        \end{itemize}
    \end{example}
\end{frame}

\begin{frame}{block}
    \begin{block}{This is a block}
        普通のブロック
    \end{block}
    \begin{alertblock}{This is a alert block}
        アラートブロック
    \end{alertblock}
    \begin{exampleblock}{This is a example block}
        example
        $r \neq 0$
        \[ \mathcal{S}' = \{ p \in \mathcal{S} \mid \min_{k_i \in \mathcal{K}} \|p - k_i\| < r_{kpp} \} \]
    \end{exampleblock}
\end{frame}
\end{document}
